\chapter{First Assembly}
\pagecolor{white}
\label{chap:31}
\begin{fullwidth}

{\itshape\bfseries “23. Keep track of every day the date emblazoned in yr morning”}

- Jack Kerouac, Belief and Technique for Modern Prose
\vspace{\baselineskip}

\problem

{\large You need to assemble a rough cut with multiple unstitched video streams. \par}

You have just rendered the quickstitches with burnt in timecode and have to select the best parts for your edit. Should you edit with the source or the stitched footage? How should you log notes for the best 360 edit?

\solution

{\large Log notes from reviewing quickstitches. \par}

Whether viewing the dailies with the crew after each day of production or during the director-editor viewing session in a headset, always log notes with the 360 space in mind. When auditioning for the best material, consider which camera the viewer will be facing when putting the headset on. Have your log sheet ready with one row per camera. 

\imgA{1}{31/dit_logsheet}

The log sheet will evolve over the entire 360 editing workflow, so make it clean and beautiful! During ingestion, have the DIT start this sheet by adding a column for each camera, a row for each take and some notes such as “bad cam”, “false take”, “dropped cam”, etc. After organizing your camera files into take folders, update this log sheet and below each take, add as many rows as the number of cameras.

\imgA{1}{31/logsheet}

The goal of the log sheet is to track the INs and OUTs of all your selects, the cameras that need some exposure correction, the synchronization offsets, the location of files and all other notes from the team. The log sheet will be extremely helpful for the stitcher, editor, and director.

{\large Put it together, stitched + unstitched. \par}

As a rough draft, the first assembly usually will have the least amount of cuts. In 360, it’s not optimal to have a lot of fast cuts and transitions. The viewer will need slow transitions to ease into the new environments. Your assembly will contain as many video/audio tracks as the number of cameras in your rig. 

First, use the quickstitches to build an edit. This method is similar to the traditional rough cut edit. Bring all your quickstitches into Premiere, use the shortcut I for IN and O for OUT to reflect the log note’s INs and OUTs selected by the director. All quickstitches should be synced and untrimmed to reflect the same timecode as the source footage you will be editing later.

When assembling all clips in your timeline, focus on the timing of the transitions. Give the viewer enough time to adjust to the new scene. Then edit all your best clips in the order you desire. 
When satisfied with the first assembly of the quickstitches, render a low resolution preview of it or start the next phase, assembly with the source footage.

\imgA{1}{31/qs_assembly}

Assembly with the source footage will require one video track per camera and should precisely match the rough cut edit of the quickstitches. Make sure the quickstitches are properly named with the take and camera number. This will make it easy for you to locate the cameras that correspond to each clip in the timeline. Select all the cameras of each take, and sync them using the multi camera Synchronization through audio.

Bring the synced sequences of source footage to a new timeline with the settings matching the camera settings. Trim based on the INs and OUTs points of your log sheet and assemble them like the previous stitched first assembly. It’s crucial to keep the same settings as the source video to avoid any compression. 

If you shot plates or created titles and other vfx, you can easily add a video track over the source video track to create the final result you are trying to achieve even before stitching it.

\imgA{1}{31/src_assembly}

The assembly using source footage is not for preview purposes, but for exporting the EDL or XML file. The EDL file or Edit Decision List is a file that many editing softwares read in order to recreate the same exact timeline after relocating the project folder and files. The purpose of the quickstitch dailies is to review and approve the shots for the first cut. After the shots are approved, the focus can turn towards fine stitching the selects. This minimizes render time and also saves time so the stitcher fine stitches only the takes needed. The most optimal workflow for the stitching post production pipeline is quickstitch all the footage, choose selects, then fine stitch the selects.


\clearpage
\end{fullwidth}