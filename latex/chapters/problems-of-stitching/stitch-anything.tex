\chapter{Stitch Anything}
\pagecolor{white}
\label{chap:49}
\begin{fullwidth}

\problem

{\large You need to stitch together footage with each angle shot at a different time. \par}

You shot different angles or different content with the same angle using one GoPro, DSLR or RED Dragon camera. You can’t synchronize using audio or motion. 

\solution

{\large Assembling ready-to-stitch camera footage. \par}

Shooting with this method is usually for creative, quality, or financial reasons. Maybe you want to create an interesting or funny scene with the same subject in every angle performing different actions. You might want to composite super crisp high resolution cinema quality shots using one RED camera or you just want to experiment with new ways to stitch and composite.

More time will be spent on the edit, deciding what should be in each angle of the experience, visual as well as sonic. To synchronize the takes that were shot during different occasions, edit your first assembly with one video track per angle. 

\imgA{1}{49/edit}

To stitch videos with completely different video settings, the overlapping area needs to feel seamless and natural for a 360 VR experience. Stitch to defish your fisheye footage one angle at a time and blend the angles in AE. You can also create overlapping areas over your source footage in AE, render each camera and utilize Autopano’s blending algorithm.
\clearpage
\imgA{1}{49/anything}

In this case, there are 4 angles with the same subject in each angle doing different things. Edit each angle in Premiere just as in the \textbf{\nameref{chap:31}} chapter. Trim and place your clips in their respective video track timelines. Delete all empty or effect spaces between clips and render each video track for the entire assembly. When rendering your files, add the camera number for the corresponding angle. Each video should account for an angle in the field of view, with its background overlapping with the other cameras. The key is to stitch the background since the actions are happening fully within each camera. 

\imgA{1}{49/times4}

\clearpage
\end{fullwidth}