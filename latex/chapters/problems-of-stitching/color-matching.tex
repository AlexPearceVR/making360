\section{Color Matching}
\pagecolor{white}
\label{chap:32}
\begin{fullwidth}
\group{stitch}

{\itshape\bfseries “Light or luminosity is created by the way elements are juxtaposed. They become reflective and a radiance comes from putting different things together.”}

- Merce Cunningham
\vspace{\baselineskip}

\problem

{\large One camera is too bright or dark, affecting the overall blending. \par}

Pure white reflects 100\% of the light, while pure black reflects 0\% of the light. Any camera’s metering system wants to meter everything as middle gray, usually around 18\% gray. Exposure compensation is a challenge during production, as it reflects 18\% of the light that is cast upon it. This is an even bigger challenge when shooting in 360 degrees. 

When correcting exposure of a camera in post production, figure out what happened in production. Was the shot overexposed? Was white balance set to auto? Exposure compensation adjusts brightness within the existing ISO Limit. If brightness has already reached the ISO Limit in a low light environment, increasing your exposure compensation will not have any effect.

\imgA{1}{32/bad_ev}

Correcting overexposed or underexposed footage with the Exposure plugin from AE or Premiere is not the way to go. Here is a good alternative to keep the bit depth of your colors at its highest.

\clearpage
\solution

{\large Read the RGB histogram. \par}

\imgA{1}{32/histograms}

Learn how to read and understand RGB histograms. R G B, red, green, blue, these 3 primary colors make up your image. Lows, mids and highs are color ranges that correspond to your low lights, also called shadows, mids and highlights. The histogram is a representation of the distribution of the colors (or pixels) in an image. 

There are two histograms. The main color histogram shows the red, green and blue channels (the actual real data) and the one channel combined value histogram is only a simulated computed value called luminosity. Use the color histogram or select an individual channel to adjust instead of the combined histogram.

You can read an overexposed shot by comparing the red, green and blue channels, and finding one or more spikes in them. A red spike in the highlights range would mean your shot was overexposed maybe by two thirds and correcting the levels of reds would help balance all colors in the image.

{\large Level gamma .2 up or down. \par}

Exposure compensation is a parameter you need to control during shooting and production while gamma correction is for post manipulation of your image. To compensate for the exposure in post production, tweak the dynamic range of color, gamma levels, or the digital interpretation of “exposure”. 

Since exposure compensation in AE or Premiere is a linear global function, it’s safe to use the gamma levels (left image below) along with its color histograms, as these are non-linear global functions compressing the dynamic range. 

\imgB{.5}{32/usinglevels}{32/usingexposure}

In AE, bring all the source footage into one composition and align them horizontally with 5-10\% overlap over each other. You can color match or exposure match all the overlapping areas or edges with this setup. When stitching, the overlapping areas will then blend much better. The pixel colors of the edges will be easier for Autopano’s algorithm to interpret. Additional control points can be found by adjusting the gamma on a shot that is over or under exposed.

\imgA{1}{32/colormatching}

Apply the plugin “Levels” on each of your video layers and review the histogram for every layer. Take note of the spikes, which will help you understand how to accurately gamma correct.

\imgA{1}{32/levels}

Adjust the gamma’s mid level, by .2 points up (to the right) or down (to the left). Try not to adjust the individual color channels, as this distorts colors too early in the post production workflow.

\imgA{1}{32/colormatched}

Finally, render a lossless .mov file instead of compressing and rendering another mp4.


\clearpage
\end{fullwidth}