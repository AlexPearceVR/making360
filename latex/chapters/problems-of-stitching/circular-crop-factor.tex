\section{Circular Crop Factor}
\pagecolor{white}
\label{chap:43}
\begin{fullwidth}
\groupR{stitch}

\problem

{\large The circumference of your fisheye lens is leaving some blurry traces in the overall blending. \par}

A fisheye lens is designed for shooting ultra wide angles, usually 180 degrees or more. The images produced are highly distorted, giving a dynamic or abstract feel. There are two types of fisheye lens, circular and full-frame. Using a circular fisheye lens results in a circular image with black edges along the frame.

\imgA{1}{43/traces}

Photographers play with the ultra wide angle effect experimenting with artistic distortion. For 360 video, fisheye lenses valuable to the engineering of the rig, improving the results of the footage. Each individual camera has a wider field of view, increasing the overlap area between cameras. Less cameras are then needed to complete a full 360 degree stitch, so the cameras can be spaced closer to each other, reducing parallax. Keep in mind the final output resolution of the panorama may decrease with the extra overlap. 

\clearpage

\imgA{1}{43/fisheye}

If you want to shoot with subjects extremely close to the camera, fisheye lenses are the way to go. However, fisheye lenses produce fuzzy edges around the image circle and capture traces of lens flares or blueish light around the image. Autopano blends some of these artifacts as well as the black frame into the stitch sometimes. You may see black or blue blobs in the blending of the sky or ground.

\solution

{\large Crop it like it’s hot \par}

When stitching footage shot with fisheye lens, check the Circular Crop tab and set the image properties. After an initial calibration, open the stitch in APG. Look for the tool with an image icon and small info “i”. Click for the popup.

\imgA{1}{43/info}
\clearpage
Autopano will then show the frame extracted for the stitching calibration of each camera. Edit the circular crop area, leaving the black and fuzzy blue out. Crop only the crisp and clean image area by decreasing the diameter of the circle. Go through each camera one at a time. 

\imgB{.5}{43/circular1}{43/circular2}

As you can see, every single lens is different, even when they are the same make and model. Their centers will be just pixels off in the frame. Autopano will update the blending and the black will not be included in the anti ghost blending algorithm. The black and blue blobs should now disappear. Use the \textbf{\nameref{chap:41}} to fine tune if there are still traces.

\clearpage
\end{fullwidth}