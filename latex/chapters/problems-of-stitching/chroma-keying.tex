\chapter{Chroma Keying}
\pagecolor{white}
\label{chap:53}
\begin{fullwidth}

\problem

{\large You need to comp a green screen plate into a stitched equirectangular video. \par}

Creating VFX that is warped correctly in 360 space is difficult, time intensive work. Detail to attention on the sync is required if you plan on animating a keyed object from the sky to wrap all around you and then onto the floor. You may get away without changing your projection or using different warping plugins, but the result is not the best. 

\solution

{\large From Autopano to Keylight. \par}

Basic green screen footage needs to be comped in your 360 panorama. The footage can be filmed with any camera like the RED or with the same 360 rig. Shooting with the same 360 rig makes the job a lot easier. With any green screen footage, defish and reproject onto a sphere. 

Import your footage along with a black video with the same camera configuration. Then click Edit to open APG. Uncheck the black video layer to render only the green screen with the right warping for a latlong 360x180. Under preview, set all blending and weighting to None. If necessary, use the Circular Crop to scale your video up. You can also set all image properties to 0 to center your video and change the FOV value, which should help scale your footage in the 360x180 projection. Render this as a 16 bit uncompressed tiff sequence.

Import into AE and add the Keylight 1.2 plugin onto the green screen tiff sequence. If your green screen was perfectly handled during production, use the “Screen colour” color selector to key out the green in your video. Otherwise, play around with the Screen Matte parameters to adjust the green keying.

When finished, place your keyed object of the subject on the center horizon of your panorama. Scale it down to fit between the first row up and down on the proportional grid to minimize the potential for distortion.

{\large AE Polar Coordinates. \par}

You want to animate your object and have it come from the top of your panorama to the center point, where your viewer is facing. Latlong projections can get tricky, as you may not be able to control the distortion required for the object to remain undistorted when the viewer looks up in the sphere.

Here’s an easy After Effects method, similar to Patching Nadir, to handle some types of VFX and keyed object compositing. Create a square composition that is equivalent to the width of the panorama, for example, 3840. Every asset that needs to be keyed will be handled in this composition. 

Bring the composition into the panoramic 2:1 composition, with its dimensions of 3840 x 1920. Add the Polar Coordinates onto the square composition. Then transform this square comp to fit in 50\% of the 2:1 comp, either position on the upper half or lower half. This will allow you to control animations for half the screen. For this example, scale the height to 25\% and position of height to 960/2. Then change the Interpolation value of the Polar Coordinates plugin to 100\%, with Polar to Rect conversion selected.

Now, animate everything in the square precomp and the projection will update itself in the main panorama.


\clearpage
\end{fullwidth}