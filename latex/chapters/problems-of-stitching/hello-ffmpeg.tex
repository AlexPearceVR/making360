\chapter{Hello FFmpeg}
\pagecolor{white}
\label{chap:56}
\begin{fullwidth}

\problem

{\large You need a tool for compression. \par}

Whether you are just starting the ingest process and need to concatenate multiple sequences, want to combine left and right eye into a well scaled over under, or want to render multiple compression tests, FFmpeg is the perfect tool for all. At any step of the 360 workflow, understanding FFmpeg will come in handy.

\solutions

{\large Hello, FFmpeg! \par}

FFmpeg is the leading multimedia framework, able to decode, encode, transcode, mux, demux, stream, filter and play pretty much anything that humans and machines have created. It supports the most obscure ancient formats up to the cutting edge. No matter if they were designed by some standards committee, the community or a corporation. - FFMPEG

{\large Install the right binary (Mac only) \par}

FFmpeg has always been a very experimental and developer-driven project. New features are added constantly and the default PC and Mac binaries may not have the needed filters or H.265 codec enabled. 

For Mac users, the right binary to install is the version 2.7.2 built by Helmut Tessarek. Unfortunately, it is compressed with 7-zip, so you may need to get a decompressor first. Use Keka (not open source, but free).

Once you unzip the binary with Keka, open your Terminal applications simply by using the search spotlight. In the terminal, you will first need to show all the hidden files and folders on your Mac, enter this line and press return:

\code{defaults write com.apple.finder AppleShowAllFiles YES}

Use the force quit menu under the Apple icon to relaunch the Finder. The hidden files should now appear in your finder window. 

You will need to place the unzipped FFmpeg binary, which should be a 30mb file, into the shared among users folder containing all your other binaries. Using the finder, Go to the Computer folder (Macintosh HD unless it was renamed) or Command + Shift + C. The folder to place the FFmpeg binary should be located at usr/local/bin, and if the folder doesn’t exist, create it. Authenticate by inputting your password. 

Now, back to your terminal window, simply using the arrow up key will bring back the line to show all hidden folders, then replace YES with NO to hide them again. Relaunch the finder. In the terminal window, simply type ‘ffmpeg’ and you should now see the right version of the binary installed.

{\large Convert Files \par}

FFmpeg is the best tool to quickly convert video and audio files to almost any formats. For example, type this line in the terminal to convert an .mov file into an .mp4 file:

\code{ffmpeg -i video.mov video.mp4}

Now let’s convert a tiff sequence into an .mp4 file. You will first need to replace the logical order of your tiff numbers with \_05\%d instead of five zeros 00000. Then, an image sequence will require a frame rate to be converted into a movie file using the -r option. Try this on a tiff sequence:

\code{ffmpeg -i sequencename_05\%d.tiff -r 25 sequence.mp4}

{\large Concatenate Sequences \par}

When recording video with the GoPro HERO4 Black, notice the files are cut off anytime after 3.8 GB. Depending on the video mode, this could be 8-11 minutes. The take will be split into chunks. This is because the microSD cards are formatted FAT32 which limits up to 4 GB. If you are using a 64 GB card, it will be formatted exFAT which normally allows larger file sizes. However, the GoPros will still limit the file size to 4GB. You can shoot a take as long as you want until the cards fill up or the battery dies. The individual segments will need to be concatenated into one file before stitching.

Use FFmpeg to concatenate your sequences. It will be much faster than using a video software’s rendering engine, like importing into AE and exporting.

Create a file mylist.txt with all the files you want to have concatenated in the following form (lines starting with a # are ignored):

\code{
file '/path/to/video1.mp4'
file '/path/to/video2.mp4'
file '/path/to/video3.mp4'
}

Note that these can be either relative or absolute paths. Then you can stream copy or re-encode your files:

\code{ffmpeg -f concat -i mylist.txt -c copy output.mp4}

{\large Combine videos into Over/Under or Side/Side \par}

If you need to combine your videos into an over under, left eye on top of the right eye, here’s the one line to enter in your Terminal after changing the filenames to match yours:

\code{ffmpeg -i left.mp4 -vf "[in] pad=iw:2*ih [left]; movie=right.mp4 [right];[left][right] overlay=0:main_h/2 [out]" output.mp4}

And for Side by Side:

\code{ffmpeg -i left.mp4 -vf "[in] pad=2*iw:ih [left]; movie=right.mp4 [right];[left][right] overlay=main_w/2:0 [out]" output.mp4}


{\large Map multiple Audio Tracks to a Video \par}

If your tiff sequence or video has no audio, you could use the default stream selection parameter -i, to add one audio track to your video: 

\code{ffmpeg -i sequence_05\%d.tiff -i audio.mp3 -codec copy -shortest output.mp4}

When you need to add multiple audio tracks to a video, for later use with head tracking for example, then you will need to understand the -map option.

\code{ffmpeg -i video.mp4 -i audio1.mp3 -i audio2.mp3 -map 0:v -map 1:a -map 2:a -codec copy output.mp4}

0:v – The 0 refers to the first input which is video.mp4. The v means "select video stream type".

0:a:0 – The 0 refers to the first input which is video.mp4. The a means "select audio stream type". The last 0 refers to the first audio stream from this input. If only 0:a is used, then all video streams would be mapped.

1:a – The 1 refers to the second input which is audio.mp3. The a means "select audio stream type".

-codec copy will stream copy (re-mux) instead of encode. If you need a specific audio codec, you should specify -c:v copy (to keep the video) and then, for example, -c:a libmp3lame to re-encode the audio stream to MP3.

-shortest will end the output when the shortest input ends.

{\large H.264 vs H.265 \par}

Apple’s current preferred compression format, H.264, has been a huge success as being most flexible codec widely used for streaming videos. Capable of handling stereo 3D videos, 48-60 FPS and even 4K resolution. The problem with H.264, however, is that while it can handle these types of encodes, it can’t do so while simultaneously keeping file sizes low. A new standard is necessary to push file/stream sizes back down while driving next-generation adoption, and that’s where H.265 comes in. It’s designed to utilize substantially less bandwidth thanks to advanced encoding techniques and a more sophisticated encode/decode model.

In order to obtain a copy of ffmpeg with libx265 support, you need to build it yourself, adding the --enable-libx265 configuration flag, with x265 being installed on your system. Here’s how you would convert and compress a tiff sequence using H.265 enabled binary:

H.264:

\code{ffmpeg -r 29.97 -i sequence_05\%d.tiff -i audio.mp3 -c:v libx264 -preset fast -maxrate 20000k -bufsize 20000k -vf scale=3840:1920 -pix_fmt yuv420p -crf 18 output.mp4}

For H.265, options are passed to x265 with the -x265-params argument such as:

\code{ffmpeg -r 29.97 -i sequence_05\%d.tiff -i audio.mp3 -c:v libx265 -preset fast -maxrate 20000k -bufsize 20000k -vf scale=3840:1920 -pix_fmt yuv420p -x265-params crf=18 output.mp4}


\clearpage
\end{fullwidth}