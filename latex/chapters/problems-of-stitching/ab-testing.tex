\chapter{A/B Testing}
\pagecolor{white}
\label{chap:46}
\begin{fullwidth}

\problem

{\large You need to compare between renders and decide what improvements are required. \par}

You can render multiple versions of the stitched video in AVP by switching .pano templates or adjusting certain settings, such as ISO or Smart blending. Before rendering your final high quality uncompressed tiff sequence, do an A/B test of the different render choices by viewing them in a headset. Seams and errors are more apparent when viewing through a headset. How do you upload, playback, and compare versions?

\solutions

{\large Playback on the desktop using Kolor Eyes. \par}

\imgA{1}{46/koloreyes}

The easiest way to quickly preview your stitch is to drop it into Kolor Eyes player for desktop. Open the application and drag your stitched mp4 or mov into the Kolor Eyes window. Use your mouse cursor to rotate around the panorama, checking for seams and any areas that require clean up.  

{\large Kolor eyes to Oculus Rift DK1 or DK2. \par}

After loading your video into Kolor Eyes, plug your Oculus Rift device into your machine with the HDMI to USB cable.

Download Oculus Runtime for Win or Mac. After the installation, go to your Applications folder and open the RiftConfigUtil.app. You should now see the Rift device recognized.

\imgA{1}{46/oculusconfig}

Next, check your Display preferences and make sure to rotate the Rift display by 90 degrees.

\imgA{1}{46/display}

Back in Kolor Eyes, you should see the eye or Oculus icon at the bottom of the window. Click on it and check in the Rift to see if your video is showing. The Rift’s headtracking sensor should now be controlling the orientation in Kolor Eyes. Put the headset on and check for seams and areas that require additional attention.

\imgA{1}{46/kolor2rift}
\clearpage
{\large Upload to YouTube 360, View with Cardboard. \par}

To preview your stitched panorama, upload your video to YouTube using their temporary \textbf{\href{https://www.youtube.com/watch?v=Z8VlD2rtACA}{Python script}}.

After uploading, go to your video page on YouTube and use your mouse cursor to rotate and check all areas for seams that need more work.

Now that your video is on YouTube, download the YouTube app onto your smartphone. For iPhone users, you can only view the 360 video with your phone and not Google Cardboard, until support between YouTube and iPhone is provided.

If you own an Android smartphone and have downloaded the YouTube app, go to your uploaded 360 video and press the “cardboard’ icon. Put your smartphone in your cardboard and check for seams that require attention.

{\large Gear VR and Note4 or Galaxy S6. \par}

If you are on a Mac, download \textbf{\href{https://www.dropbox.com/sh/8kqte3wtyt3vltu/AAAjUnNxtXhaxprWF8_q9zbCa?dl=0}{SmartSwitch}} to upload files to a Samsung device. Install the application and let it restart your computer.

Start SmartSwitch and click on the dropdown arrow next to the phone name. You will then see a ‘folder’ icon to open your phone's library. In this directory, go to the "Oculus" folder and create a "360Videos" folder. Drag any of your 360 videos to test into this folder.

\tip For stereoscopic 3D 360 uploads, make sure to add "\_TB" for Top / Bottom or Over Under files, and "\_LR" for Left / Right or Side by Side files.

\imgB{.5}{46/smartswitch}{46/internaldrive}

Under ‘Apps’, launch the Oculus app and start the 360 Videos app. Put the headset on to review. You can also access the 360 Videos app from the Oculus Home menu.

\imgA{1}{46/oculushome}

Tap the touchpad of your Gear VR while pointing the gaze cursor on your video. Then check for the seams that require your attention.

\clearpage
\end{fullwidth}