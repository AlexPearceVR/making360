\section{Formatting Cards}
\pagecolor{white}
\label{chap:4}
\begin{fullwidth}
\groupL{setup}

{\itshape\bfseries “N.Z: I suppose your explorations of new media are like swimming in an endless ocean. 

N.J.Paik: A tabula rasa, you know a white paper. Video is a white paper, a tabula rasa.”}

\vspace{\baselineskip}

\problem

{\large How do you keep track of all the cameras and tiny microSD cards? \par}

Be organized! Number your cards as well as cameras. Color code your cameras if you have multiple rigs. This will prevent headaches and confusion during textbf{\nameref{chap:29}} and post production. There are all the normal problems of shooting times x amount of cameras so proceed with extra care. 


\solution

{\large Blank canvas and tweezers. \par}

Before every shoot, format all your cards or file management will get really messy. Keep the same microSD card per camera so it is easier to troubleshoot. For example, if one card has corrupted files, footage that is out of focus, over exposure, or other problems you can track it down to the exact camera. Of course, always double check that your footage has been backed up before formatting. 

Formatting the cards through the camera is best instead of on the computer so the original file structure and partitions are restored. 

\imgA{1}{4/cam1_3}
\imgA{1}{4/cam4_6}
\imgA{1}{4/sd_numbered}

\tip For the non-tiny hands, tweezers are very useful for getting the microSD cards in and out of the cameras especially when they are in the rigs and less accessible. 



\clearpage
\end{fullwidth}
