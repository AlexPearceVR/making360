\chapter{Modified Fisheye}
\pagecolor{white}
\label{chap:26}
\begin{fullwidth}

\problem

{\large You want to shoot with fisheye lens.\par}


\solution

{\large Carefully remove and replace the lens. \par}

With modified fisheye lens on the GoPros, you have greater coverage per camera with more overlap. Less cameras are needed in the rig for a full 360 stitch which means less seams and parallax! Subjects can get real close to a camera without breaking a seam. With more overlap, you can also shift the seams when rotoscoping or masking. However, since there is so much more coverage, a lot of the image is in the overlap, resulting in a lower final output resolution for the panorama. Shoot on the 2.7K settings to achieve 4K final output. 

\imgA{1}{26/overlap}

Step one. Prepare your tools. Removing a GoPro lens is a very meticulous process and you surely don't want to have to tear it down. Here are the tools you need. 

GoPro HERO3 or 4
\\
Lens
\\
Lenscollar
\\
Flat screwdriver
\\
Mighty wrench
\\
Heat gun

IMAGES

Step two. Remove the battery and SD card.

IMAGE

Step three. Remove the lens outer ring using the srewdriver from the three different positions. You will be breaking the glue points, pulling out the outer ring gently.

IMAGE

Step four and most critical step. Remove the lens. The lens has to be unscrewed from the GoPro lens mount. Since the original lens is also glued from the inside of the lens mount, use the heat gun for 5-10 seconds over and around the lens. Use the mighty wrench to hold the lens strongly while rotating counter-clockwise the GoPro body. Keep rotating it until you are able to unscrew the lens with your fingers. If need be use the heat gun again instead of forcing.

IMAGE

Step five. Clean up the glue from the lens mount using your screw driver.

Step six. Add the lens collar to your new fisheye lens.

Step seven. Insert and screw the new fisheye to the body of your GoPro.

Step eight. Put the battery in and connect your GoPro to a monitor using an HDMI mini to HDMI cable.

Step nine. Calibrate the focus by holding a checkerboard print in front of the lens, and screwing the lens until perfect focus is achieved. 

IMAGE

Step ten. Lock the focus by screwing the lens collar with the flat screwdriver and until the lens is unlocked and unscrewable by hand.

IMAGE


\clearpage
\end{fullwidth}
