\chapter{Pair to Remote}
\pagecolor{white}
\label{chap:09}
\begin{fullwidth}

\problem

{\large The camera rig is out of reach and you can’t manually hit record.  \par}

In some situations, you won’t be able to manually trigger the cameras. For example, if the cameras are rigged high up, on a dolly or drone. You can use a WIFI or smart remote to easily turn all the cameras in a rig on and off!  


\solution

{\large Use a WIFI or Smart remote. 
 \par}

The remotes can trigger up to 50 gopros at once. Before the shoot, pair a remote to the rig. The WIFI will have to be turned on for each camera. This will drain the batteries faster so if you are in a situation where you can’t constantly charge batteries and backups, then make sure to save power by turning the WIFI off between takes. 

To pair a remote to your GoPro HERO4, turn the camera on and enter the wireless menu. Select “REM CTRL” and select “NEW” pairing. The camera will be in pairing mode for 3 minutes.

\imgA{1}{9/goproinpair}

Next, turn the remote on and put into pairing mode. If you have an older WIFI remote, hold down the red shutter button and press the white power button to turn the remote on and enter pairing mode. If you have a Smart remote, turn it on with the power and mode button. Once it shows a WIFI symbol, hold down the Settings/Tag button to enter pairing mode. 

\imgA{1}{9/remoteinpair}

Both the camera and remote should now have two arrows pointing towards each other in the display. The remote will ask you if you want to pair another camera. Repeat until you have all your cameras connected.

\imgA{1}{9/paired}

You will probably want to use this method even if the camera is in reach so the cameras will roll at approximately the same time. The cameras will still be a few frames off from each other.  If you are on set or have backup batteries then leaving WIFI on will not be an issue. If you are in the field and need to save all the power you can, stick to manually triggering the cameras. 



\clearpage
\end{fullwidth}
