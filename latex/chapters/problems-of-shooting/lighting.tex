\chapter{Lighting}
\pagecolor{white}
\label{chap:25}
\begin{fullwidth}


\problem

{\large You want the most optimal lighting conditions for the shoot. \par}

Lighting is tricky with the GoPros. With low lighting conditions, the image has a lot of noise. Too much artificial lighting causes a variety of problems such as a blown out image, pollution, and color variation. Lens flares are also more common with wide angle and fisheye lens. If shooting in stereo, the flares and differences cause a jarring image. Also, where do you hide the lights??

\solution

{\large Stay natural. \par}

When shooting outdoors, if you have the time and patience, wait for the right moment. Try to shoot after before dusk when the sun has just set but still emits a light hue. However, there is a small window to catch the perfect timing. If shooting at a different time of day, the sun will be pointing directly into one of the cameras, causing overexposure. You can shoot the moon and patch the nadir if it is the camera pointing towards the sky. 

For interior shots, do not use too many different artificial lights. They will cause various colors and shadows. Unless you are shooting a JJ Abrams style VR piece, be careful with pointing light sources directly and the lens, creating lens flares.

Be careful not to use too much tungsten lighting or the infrared pollution will cause a purplish hue and need to be color corrected. 

Everything will show in a 360 shot and unfortunately you cannot hide and seek with lights. Try dressing your set and hiding light sources in blind spots. 

{\large Get some outside help. \par}

If you are filming on location at a physical structure like a house, a low rise building, or a church you have a nice option for adding some extra light to your interior 360 shots without having to do the usual post production and roto cleanup work to hide the light fixtures that would be visible if they were placed inside the actual room you are filming. 

You can do this by renting a few large HMI lights from a grip & camera supply company. If you place the HMI light fixtures outside the structure you can beam more light into your interior shots through the windows. HMI lights have a nice property where they give a very clean and high output illumination that closely matches the natural color temperture of daylight.

One thing to keep in mind is that since HMI lights use a metal halide gas, have a very high power demand like 12 kilowatts / 18 kilowatts / or 24 kilowatts that typically require a generator, and get hot quickly you need to make sure to get some help from a professional lighting crew person when moving and setting up the lighting gear so you don't have issues or loose valueable time during a shoot.

Also keep the HMI lights set back a distance from any wooden siding or flammable objects on the location as the lights and objects near by can get warm if the lights are left on for a few hours. Also, when renting gear, it is good to know that using a modern HMI light fixture is much better than renting an older model as the new HMI light controllers can hot restrike the arc on the HMI gas so you don't have as much down time between turning the lights on and off.


Popular HMI lights for on location lighting use are the ARRI \textbf{\href{http://www.arri.com/lighting/lighting_equipment/lampheads/daylight_hmi_lampheads/}{ARRISUN}}, or Mole Richardson's \textbf{\href{http://mole.com/products/index/daylite-fesnels}{Daylite Fresnels}}.

NEED CONTRIBUTORS

\clearpage
\end{fullwidth}