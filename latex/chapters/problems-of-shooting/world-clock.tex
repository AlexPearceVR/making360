\chapter{World Clock}
\pagecolor{white}
\label{chap:5}
\begin{fullwidth}


\problem

{\large You need a system for file naming convention. \par}

Camera 1 files start at GOPR0001.mp4, Camera 2 at GOPR1234.mp4, Camera 3 at GOPR4747.mp4 and so on. 


\solution

{\large Use Time/Date for naming convention. \par}

Synchronize the clocks for all cameras. This will make file management and comparing takes much easier later on. Inside each take folder you can check the details section and confirm that all the videos in that take start at the same time. 

To set the Time/Date, use the menus on the GoPro or connect the camera to the GoPro app or software and you can manually set the clocks.

IMAGE of GoPro in Time/Date.

Here the clocks were not set. To confirm the videos are all from the same take, you can check the file size. 

\imgA{1}{5/worldclock1}

Here the clocks were set and much easier for the DIT or stitcher to organize the files.

\imgA{1}{5/worldclock2}

For the month/day, you can use the day as the camera number and month if there are multiple rigs. 

January 01 - camera 01, rig 1
\\
January 02, camera 02, rig 1
\\
February 04, camera 04, rig 2


The Time/Date can be used as a form of metadata for reference during the textbf{\nameref{chap:29}} and stitching process later on. 

\tip Every time you update the firmware for GoPros or if you leave the battery out for an extended time, the Time/Date will reset. Make sure to go back and synchronize the clock for all the cameras. Although you will be able to tell if the files are part of the same take from other details, like file size, it is easier to pull files across all cameras into a take folder from the same start time. 




\clearpage
\end{fullwidth}
