\chapter{Platonic Rigs}
\pagecolor{white}
\label{chap:2}
\begin{fullwidth}

{\itshape\bfseries “There is geometry in the humming of the strings, there is music in the spacing of the spheres.”}

- Pythagoras
\vspace{\baselineskip}

\problem

{\large You need to choose a 360 camera rig from all the options and configurations available. \par}

The popular 6 camera cube? 7 camera cylindrical layout? 10 camera layout? Or pehraps 3 cameras with modified fisheye lenses? Mono or stereo? What about spherical or cylindrical? One size does not fit all. Don’t worry, we’ll find the perfect fit. Selecting a rig depends on the type of content you are shooting, environment, distance, moving shots and of course money in the piggy bank.

\solution

{\large Prioritize your needs. \par}

{\bfseries MONO vs STEREO}

\imgA{1}{2/mono_stereo}
\clearpage
First decide between using a monoscopic or stereoscopic 360 video rig. 

With a mono 360 rig, all of the cameras will together capture a single 360 video. No illusion of depth can be achieved. This is the simplier approach in every way.

But a stereo 360 rig is specially designed to have cameras for the left and right eyes. Hence the need for double the amount of cameras. In this way 360 video can be achieved in 3D. But there are a few caveats...

In the end, it all comes down to overall cost. Stereo will give depth to the subjects and objects, enhancing the quality of experience but the costs will be significantly greater in both hardware and post production. If you have the budget and manpower, then stereo for the win! The difference is astounding and makes the experience more vivid and real. 

If you are shooting on a tight budget look at the type of content you are shooting. Will the subjects and objects be close? If they are at farther distances or if you are shooting landscapes without close subjects, the stereo effect won’t be as noticeable. You should save your money for a different aspect of the production. Another factor to keep in mind is how much control you have over the environment. If you are shooting a live event like news or sports then it will be difficult to control variables like subjects moving between seams of the cameras. The parallax and flaws between cameras is even more apparent especially for differences between stereo pairs. The errors will exponentially compound and cause viewing discomfort, eyestrain, and nausea. Shooting on set where you can control variables and block movement will be best for stereo. If you are sending a stereo rig into the field, be prepared for potentially heavy post production since environment variables will be out of your control. 

With monoscopic videos you will be able to get higher resolution. To playback stereoscopic videos the left and right eye videos are stacked over/under and combined into one file resulting in half the resolution. If you don’t have the budget for stereo, don’t be too bummed as you can capture more detail over stereo with 4K, 8K, and even 12K resolution out of your mono videos!

{\bfseries SPHERICAL vs CYLINDRICAL}

\imgA{1}{2/spehrical_cylindrical}

If you have decided to stay mono, there are quite a range of options for you to choose from that offer high resolution. Again, pick the rig based on the style and type of content you are shooting. If you are shooting landscape with minimal subjects, then a cylindrical rig with more cameras around will offer extra high resolution. There will be more camera coverage around the horizon. However, because of the limited vertical FOV (field of view), there will be a hole at the nadir (floor) or zenith (sky). In other words, there will a zone where footage is not captured. But this may be ok because the viewer will not be looking at the sky or floor most of the time. So if you are shooting for a dome, the nadir hole won’t be a problem since the camera rig will be on a tripod and won't be rendered into the fisheye shot.

The sky and floor can also be shot with an extra camera. You can even use a still camera, such as a Nikon or Canon. Then during the stitching process fix the missing zone and patch in the nadir hole or replace the tripod. 

But A cylindrical rig is not ideal if you have multiple subjects moving around between cameras. More money, more problems. More cameras, more seams!

A spherical hemicube rig is an option if you have a smaller budget and less cameras on hand. There will be equal coverage between the cameras including the zenith and nadir.
\clearpage

{\bfseries FISHEYE vs WIDE-ANGLE}

\imgA{1}{2/fisheye_wide}

Another option for rigs is to modify the camera with fisheye lens. You can achieve a greater FOV than wide angle lens and have more coverage per camera. In effect you will need less cameras for the rig, which allows the cameras to be closer together and have less parallax. An advantage of this rig is allowing subjects to get up close to the camera because there are less cameras and seam lines to break. It also allows for more footage overlap and can really help to hide any seams during the stitching process.

{\bfseries RECOMMENDED MODELS}

MONO 
\begin{itemize}
\item 6 hemicube camera rig
\item 10 camera rig
\item 3 camera modified fisheye rig
\item 4 camera modified fisheye rig 
\end{itemize}

STEREO 
\begin{itemize}
\item 12 camera rig
\item 14 camera rig
\item 6 camera modified fisheye rig
\item 8 camera modified fisheye rig
\end{itemize}
\clearpage

For a comprehensive list of existing available solutions, see Jason Fletcher's collection of 360 video rigs on The Fulldome Blog.
            
MODELS OF THE UNIVERSE

\imgA{1}{2/models}
                
Find the balance between the FACTORS
\\
TIME...is money
\\
MONEY...is power
\\
DISTANCE...is time apart 
\\
DEPTH...is love
\\
CONTROL…is an illusion
\\
RESOLUTION...is a state of mind


\clearpage\end{CJK*}

\clearpage
\end{fullwidth}
