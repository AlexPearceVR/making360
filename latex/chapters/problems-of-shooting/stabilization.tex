\chapter{Stabilization}
\pagecolor{white}
\label{chap:24}
\begin{fullwidth}


\problem

{\large You are shooting a moving shot and need to have the smoothest stabilization. \par}

The easiest type of 360 video shots are static, with the rig placed in a fixed position. But adding smooth camera movement can add a great sense of immersion to a shot. After all, as humans we are always moving and it's what we expect when watching media, otherwise it can feel a bit dull. Dolly and drone shots add exciting movement to the experience. However, extreme care and caution in stabilizing the shot is needed or the viewer may get instant motion sickness. Any movement of the camera is magnified in a VR headset and can cause nausea if not shot properly.


\solution

The biggest thing to keep in mind is the inherent limitations of your specific rig. This isn't a typical camera... And so using it requires a different frame of mind. Indeed, not only does the technology have many hurdles but also the process of shooting a moving shot!

While recording the shot if any actors or objects get too close to the rig, on average 6 feet, then the parallax seams are going to be very obivious upon stitching. These seams completely ruin immersion because it suddenly exposes the magic. So create an imaginary boundary line in your mind and your shots will be beautiful and your stitching easier.

Each rig is different and you’ll have to experiment to understand its particular limits of where the parallax is too obvious. Such as for rigs which use fisheye lenses, they have a huge amount of footage overlap so you can get closer to the rig.

Prior to the 360 shoot, think through the camera movements in-depth. Make sure that everyone involved in the shoot understands where the imaginary boundary line is and why they shouldn't get too close to the rig.

{\large A few crazy but surprisingly effective solutions \par}
Most solutions for obtaining a super smooth camera motion will mean that the dolly, drone, or vehicle will be within the shot. So get creative and figure it out! Here are a few tried and true approaches:

{\bf 3-axis Gyroscope}
\\
On one end of a monopod stick is your 360 rig, and on the other end is the gyroscope. There is a sweet spot on the monopod stick where you can hold it and the 360 rig will remain high up and level, but the gyro will keep the shot extremely smooth. It is an odd sensation to walk around with it since gyro's can act strangely when you turn certain directions, but that's just part of the technique. But there are restrictions in their use. They are heavy, weighing between 2-5kg (4-10lbs). Also, they take about 8-10 minutes to spin up and then their run time is limited. The Kenyon Gyro Stabilizers are pretty incredible.

{\bf Tracked Dolly}
\\
So you lay down some rail road tracks and then put your tripod onto it. Then you can crawl on the ground and push your rig along very slowly. Or you can use some rope and pull it slowly from a distance. But yes, the dolly tracks will be in the shot; perhaps you can find a way to hide them? The Singleman Indie-Dolly is surprisingly affordable and is very portable. 

{\bf Drone UAV}
\\
This can be tricky since the drone needs to be able to lift the weight of the 360 camera rig, but there are options out there. There is even a custom and bizarre rig which makes the drone itself \textbf{\href{youtu.be/-5iUZybKXr0}{invisible}}.

{\bf Fishing Line}
\\
Grab some high gauge fishing line and string it between two points high up, while making sure one point is slightly lower. This will ensure that the 360 rig will slide down the incline. Be sure to first test it out with some object that matches the weight of your 360 rig... The last thing you want is for your 360 rig to come crashing down!

{\bf Manual / Motorized Wheelchair}
\\
A manual wheelchair can provide a surprisingly smooth shot. The tricky thing here is to make sure that you are pushing the wheelchair evenly forward. It's easy to accidentally get some shimmy motion, since its difficult to make it just move perfectly forward. But this is obviously not a problem with a motorized wheelchair. 

{\bf Car, Golf Cart, Airplane}
\\
Use powerful suction cups to mount the 360 rig to anywhere. Cinetics makes a wonderful mount called the CineSquid to do just this.





\clearpage
\end{fullwidth}
