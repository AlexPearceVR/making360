\chapter{Stabilization}
\pagecolor{white}
\label{chap:24}
\begin{fullwidth}


\problem

{\large You are shooting a moving shot and need to have the smoothest stabilization. \par}

Most 360 video shots are static, with the rig placed in a position. But adding smooth camera movement can bring a great sense of immersion to a shot. After all, as humans we are always moving and its what we expect when watching media. A still shot can sometimes feel a bit flat. Dolly and drone shots add exciting movement to the experience. However, extreme care and caution in stabilizing the shot is needed or the viewer may get instant motion sickness. Any movement of the camera is magnified in the headset and can cause nausea if not shot properly.


\solution

Most solutions for obtaining a super smooth camera motion will mean that the dolly, drone, or vehicle will be within the shot. So get creative and figure it out!

{\large A few crazy but surprising effective solutions \par}
{\bf 3-axis Gyroscope} - On one end of a monopod stick is your 360 rig, and on the other end is the gyroscope. There is a sweet spot on the monopod stick where you can hold it and the 360 rig will remain high up and level, but the gyro will keep the shot extremely smooth. It is an odd sensation to walk around with it since gyro's can act strangly when you turn certain directions, but thats just part of the technique. But there are restrictions in their use. They are heavy, weighing between 2-5kg (4-10lbs). Also they take about 8-10 minutes to spin up and then their run time is limited. The Kenyon Gyro Stabilizers are pretty incredible.

{\bf Tracked Dolly} - So you lay down some rail road tracks and then put your tripod onto it. Then you can crawl on the ground and push your rig along very slowly. Or you can use some rope and pull it slowly from a distance. But yes, the dolly tracks will be in the shot; perhaps you can find a way to hide them? The Singleman Indie-Dolly is surprisingly affordable and is very portable. 

{\bf Drone UAV} - This can be tricky since the drone needs to be able to lift the weight of the 360 camera rig, but there are options out there. There is even a custom and bizarre rig which makes the drone itself invisible: youtu.be/-5iUZybKXr0

{\bf Fishing Line} - Grab some high gauge fishing line and string it between two point high up, while making sure one point is slightly lower. This will ensure that the 360 rig will slide down the incline. Be sure to first test it out with some object that matches the weight of your 360 rig... The last thing you want is for your 360 rig to come crashing down!

{\bf Manual / Motorized Wheelchair} - A manual wheelchair can provide a surprisingly smooth shot. The tricky thing here is to make sure that you are pushing the wheelchair evenly forward. Its easy to accidentally get some shimmy motion, since its difficult to make ot just move perfectly forward. But this is obiviosuly not a problem with a motoized wheelchair. 

{\bf Car, Golf Cart, Airplane} - Use powerful suction cups to mount the 360 rig to anywhere. Cinetics makes a wonderful mount called the CineSquid to do just this.





\clearpage
\end{fullwidth}
