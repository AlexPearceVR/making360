\chapter{Set and Settings}
\pagecolor{white}
\label{chap:6}
\begin{fullwidth}


{\itshape\bfseries The nature of the experience depends almost entirely on set and setting. Set denotes the preparation of the individual, including his personality structure and his mood at the time. Setting is physical — the weather, the room’s atmosphere; social — feelings of persons present towards one another; and cultural — prevailing views as to what is real. It is for this reason that manuals or guide-books are necessary. Their purpose is to enable a person to understand the new realities of the expanded consciousness, to serve as road maps for new interior territories which modern science has made accessible.}

-Timothy Leary, The Psychedelic Experience: A Manual Based on the Tibetan Book of the Dead
\vspace{\baselineskip}

\problem

{\large You have to set all the settings on the cameras. \par}

You will have to manually set each camera by hand so decide the default settings you want to shoot before changing them. Every camera must have all the same settings, especially frame rate!

\solution

{\large Keep it RAW. Match all the cameras. Find the sweet spot between resolution and framerate. \par}

You want your settings matched identically across all the cameras. This will allow them to stitch better and have less color matching and balance to correct in post. Start by deciding the framerate and aspect ratio. This depends on the rig you selected. Certain rigs require a 4:3 aspect ratio instead of 16:9 so there is enough overlap between all the cameras to stitch. 

If the cameras accidentally get knocked and the settings change it is ok as long as the framerate and aspect ratio stayed the same. Even if one of the aspect ratios was different you might still be able to salvage the shot with some serious warping of that one camera. 

Or if the exposure is drastically different in some of the cameras you can do some color correcting. However, if one of the camera’s frame rate changes you will be out of luck! There needs to be the same number of frames for the stitching software to apply a calibration to. 

{\large Protune - on \par}

\imgA{1.5}{6/gopro_protune}

The protune setting should always be kept on. Protune will give you much higher dynamic image range and overall image quality with more detail in highlights and shadows. The image will shoot flatter for more freedom in color correction.  Protune has higher data rate capture (up to 60 mbps) and less compression, giving you more information to work with. Having a neutral color profile across all the cameras will make them easier to color balance and correct for a nice stitch. 

\tip Turn protune ON first before you select all the other settings because all the settingsfor resolution and fps reset when protune is changed.

\clearpage
{\large White Balance - cam raw \par} 

\imgA{1.5}{6/gopro_wb}

This keeps the color flat but you keep more information which you can color correct and grade during post production.

{\large Resolution/FPS \par}

Next decide your aspect ratio. Depending on which rig you are using, certain settings must be used for there to be enough overlap between the seams. 

For a hemicube rig like Freedom360, 360 Heros Pro6, or 360Abyss the aspect ratio has to be 4:3 so there is enough overlap in the seams. 

The most recent GoPro Hero 4s now offer:

2704x2028 at 30fps

1440x1920 at 80fps 

\clearpage
1280x960 at 100fps 

\imgA{1.5}{6/gopro_960_100fps}

For more cylindrical rigs, the aspect ratio can be 16:9 because each camera will be closer to the adjacent left/right camera.The 16:9 aspect ratio will offer enough overlap. Then you can use the 2.7k settings and have a higher resolution output stitch like 8k. 

2704x1520 at 60fps

\imgA{1.5}{6/gopro_27_60fps}

\clearpage
1920x1080 at 120fps

\imgA{1.5}{6/gopro_1080_120fps}

Choosing a higher frame rate will sacrifice resolution. Shoot at a higher fps for fast high action scenarios like drone shots or underwater. Higher frame rate gives more frames to sync the cameras.

{\large ISO limit - 400 \par}

\imgA{1.5}{6/gopro_iso}

This adjusts the camera’s sensitivity in low light conditions. Keep it at 400 which will give you darker videos but the least noise and gain. 

\clearpage
{\large Low Light - off \par}

\imgA{1.5}{6/gopro_lowlight}

The camera will automatically adjust to changes in exposure when shooting in low light environments. Again, any setting where the cameras are automatically changing we want off so the cameras stay the closest settings to each other. 

{\large Spot Meter - off \par}

\imgA{1.5}{6/gopro_spotmeter}

\clearpage
{\large Sharpness - low \par}

\imgA{1.5}{6/gopro_sharpness}

The videos will need to be sharpened during post production for more clarity and details in the headset. Use the low setting for less processing on the footage and more data in post. 

{\large Exposure compensation - 0.0 \par} 

\imgA{1.5}{6/gopro_evcomp}

Range is -2 to +2, in 0.5 step increments. Leave the exposure on 0.0 and equal on all the cameras. If you have one or two cameras pointing at the sky, you can bump just those cameras up to +1.0 or +2.0. If you have realtime preview or field monitor with you, try out and adjust the settings accordingly. 

When using a new rig for a shoot, test the cameras and adjust settings the day before! Unload the footage and do a test stitch to double check and make sure the settings are correct and best for that rig. If you are torn between higher fps or resolution, do a test and check it out in the headset before. After you find the sweet spot, write down the settings and charge up the batteries for the shoot. Check again on the day of to make sure the settings did not accidentally get knocked in transit. 

\clearpage
\end{fullwidth}