\section{Safety Zones}
\pagecolor{white}
\label{chap:23}
\begin{fullwidth}
\group{plan}

{\itshape\bfseries And where do you place the viewer in all this?

“As someone seated on the bank of the gushing torrent, taking in everything that flows past, the flurries of motion and the moments of calm. But I hope also as someone who plunges into the current, literally bathes in it, carried away by the flight of their own imagination.”}

- Hou Hsiao Hsien, Interview on THE ASSASSIN
\vspace{\baselineskip}

\problem

{\large You don’t want seam lines, ghosts or broken limbs. \par}

If you don’t want siamese twins, meaning weird errors during the stitching process, then keep subjects out of the hazardous seamline zone. This will save frustration and hours of keyframing, \textbf{\nameref{chap:50}}, using \textbf{\nameref{chap:41}} and render time during post. 

Think like a stage director or magician and block actors for the space. 


\solution

{\large Block subjects to stay within boundaries. \par}

Have the subjects stay in fixed areas within a camera. If they must move between cameras, have them cross a seam line at a further distance from the rig so they are smaller. The seam will then be less noticeable. Tape, mark down, and rehearse. Remember to remove the guidelines before shooting or they will be in the shot! If you have a \textbf{\nameref{chap:11}} or field monitor, check to see if subjects are sitting in or crossing a seamline. 
\clearpage
\imgA{1}{23/inseam}
\imgA{1}{23/further}
\imgA{1}{23/close}

If you are in the field and have no control over the environment, then adjust and turn the camera rig for the least seams on the main subject and action areas. In VR, you are not chasing a shot, but setting up the rig and letting the moment flow to you. Follow your intuition and place the rig in a good position. While you can’t frame the shot, you can think spherically and compose the space. 

{\large Clean plates. \par}

If you are shooting a scene where subjects must cross seamlines, record a take of just the environment. This will give you a clean plate of the background for \textbf{\nameref{chap:50}} and stitching with the \textbf{\nameref{chap:34}} approach. 


\clearpage
\end{fullwidth}
