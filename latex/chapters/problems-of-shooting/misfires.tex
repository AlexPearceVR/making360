\section{Misfires}
\pagecolor{white}
\label{chap:20}
\begin{fullwidth}
\group{shoot}

\problem

{\large Oops! You hit record on one of the cameras by accident. \par}

It’s easy to trigger one or two of the cameras and multiple times throughout the shoot. This offsets the camera take numbers from each other and makes it a headache during \textbf{\nameref{chap:29}}. For example, after take 01 you triggered camera 1 by accident. After the next take, camera 1 will be  at 03 files but the rest of the cameras will say 02. If you triggered different cameras more than once, things start getting more confusing!

\imgA{1}{20/misfire1}
\imgA{1}{20/misfire2}

\clearpage
\solution

{\large Play catchup. \par}

If you trigger an individual camera, check the count number on all the other cameras. Then trigger those cameras for 1-2 seconds to increase the take count number. For example, if you misfired camera 01, then trigger cameras 02-06 to match the same amount of takes as camera 01 and all around. If you misfired a camera twice, then apply the same method. During the ingest, the takes will then match across all the cameras and be easier to separate into take folders.  

\imgA{1}{20/misfire3}
\imgA{1}{20/misfire4}


\clearpage
\end{fullwidth}
