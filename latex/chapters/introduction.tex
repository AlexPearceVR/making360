
%----------------------------------------------------------------------------------------
%	INTRODUCTION
%----------------------------------------------------------------------------------------
\SkipTocEntry\section*{introduction}
\addtocounter{section}{2}
%------------------------------------------------
\begin{fullwidth}

{\itshape\bfseries by Jason Fletcher

}

Let no one say otherwise, shooting 360 video is difficult and intense! It’s a medium that has completely unique challenges. And that is exciting for both the tech folk and the storytellers. But you’ll need to understand the many hurdles so that you can soar. Knowing the specific details, inherent limitations, and potential problems will only help to inform how to successfully create immersion. And that is what this book aims to do!

We are going to throw a bunch of information at you. Yet it's really up to you to connect the dots and understand the optimal workflow for your specific camera rig. This isn’t your typical DIY book. Really to become adept at 360 video, you will need to perform test shoots and run into problems yourself. The best way to learn and gain valuable experience is to fail! With that said, we will equip you with a comprehensive approach.

{\large Big Picture Workflow\par}

There are many details that we need to discuss. And so we are taking a brute force approach, organized into chapters. But in reality there are distinct steps in a typical 360 video shoot:

\begin{itemize}
\item EQUIP: choose your gear
\item SETUP: camera settings, memory cards, pair remote
\item PLAN: stabilization, safety zones, actor blocking
\item SHOOT: recording, synchronization, lighting
\item IMPORT: ingest, file management 
\item STITCH: dailies quickstitch, color matching, render tiff
\item EDIT: rotoscoping, color grading, final render
\end{itemize}

\clearpage
\end{fullwidth}
%----------------------------------------------------------------------------------------
